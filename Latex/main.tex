%%%%%%%%%%%%%%%%%%%%%%%%%%%%%%%%%%%%%%%%%
% Homework Assignment Article
% LaTeX Template
% Version 1.3.1 (ECL) (08/08/17)
%
% This template has been downloaded from:
% Overleaf
%
% Original author:
% Victor Zimmermann (zimmermann@cl.uni-heidelberg.de)
%
% License:
% CC BY-SA 4.0 (https://creativecommons.org/licenses/by-sa/4.0/)
%
%%%%%%%%%%%%%%%%%%%%%%%%%%%%%%%%%%%%%%%%%

%----------------------------------------------------------------------------------------

\documentclass[a4paper]{article} % Uses article class in A4 format

%----------------------------------------------------------------------------------------
%	FORMATTING
%----------------------------------------------------------------------------------------

\addtolength{\hoffset}{-2.25cm}
\addtolength{\textwidth}{4.5cm}
\addtolength{\voffset}{-3.25cm}
\addtolength{\textheight}{5cm}
\setlength{\parskip}{0pt}
\setlength{\parindent}{0in}

%----------------------------------------------------------------------------------------
%	PACKAGES AND OTHER DOCUMENT CONFIGURATIONS
%----------------------------------------------------------------------------------------

\usepackage{blindtext} % Package to generate dummy text
% \usepackage[style=numeric,sorting=none]{biblatex}
\usepackage{charter} % Use the Charter font
\usepackage[utf8]{inputenc} % Use UTF-8 encoding
\usepackage{microtype} % Slightly tweak font spacing for aesthetics

\usepackage[english]{babel} % Language hyphenation and typographical rules

\usepackage{amsthm, amsmath, amssymb} % Mathematical typesetting
\usepackage{float} % Improved interface for floating objects
\usepackage[final, colorlinks = true, 
            linkcolor = black, 
            citecolor = black]{hyperref} % For hyperlinks in the PDF
\usepackage{graphicx, multicol} % Enhanced support for graphics
\graphicspath{ {./images/} }
\usepackage{xcolor} % Driver-independent color extensions
\usepackage{marvosym, wasysym} % More symbols
\usepackage{rotating} % Rotation tools
\usepackage{censor} % Facilities for controlling restricted text
\usepackage{listings, style/lstlisting} % Environment for non-formatted code, !uses style file!
\usepackage{pseudocode} % Environment for specifying algorithms in a natural way
\usepackage{style/avm} % Environment for f-structures, !uses style file!
\usepackage{booktabs} % Enhances quality of tables

\usepackage{tikz-qtree} % Easy tree drawing tool
\tikzset{every tree node/.style={align=center,anchor=north},
         level distance=2cm} % Configuration for q-trees
\usepackage{style/btree} % Configuration for b-trees and b+-trees, !uses style file!

% \usepackage[backend=biber,style=numeric,
            % sorting=nyt]{biblatex} % Complete reimplementation of bibliographic facilities
% \addbibresource{ecl.bib}
\usepackage{csquotes} % Context sensitive quotation facilities

\usepackage[yyyymmdd]{datetime} % Uses YEAR-MONTH-DAY format for dates
\renewcommand{\dateseparator}{-} % Sets dateseparator to '-'

\usepackage{fancyhdr} % Headers and footers
\pagestyle{fancy} % All pages have headers and footers
\fancyhead{}\renewcommand{\headrulewidth}{0pt} % Blank out the default header
\fancyfoot[L]{School of Computing, Macquarie University} % Custom footer text
\fancyfoot[C]{} % Custom footer text
\fancyfoot[R]{\thepage} % Custom footer text

\usepackage{comment}
\newcommand{\note}[1]{\marginpar{\scriptsize \textcolor{red}{#1}}} % Enables comments in red on margin

\usepackage{hyperref}

% \usepackage{makeidx} 
% \makeindex
%----------------------------------------------------------------------------------------

\begin{document}

%----------------------------------------------------------------------------------------
%	TITLE SECTION
%----------------------------------------------------------------------------------------

\title{COMP3100 project report} % Article title
\fancyhead[C]{}
\hrule \medskip % Upper rule
\begin{minipage}{1\textwidth} % Center of title section
\centering 
\large % Title text size
Project report: Stage 1\\ % Assignment title and number
COMP3100 Distributed Systems, S2, 2022\\
\normalsize % Subtitle text size
SID: 46225854, Name: Kris Paul Babu
%%%%\\ % Assignment subtitle
\end{minipage}
\medskip\hrule % Lower rule
\bigskip

%----------------------------------------------------------------------------------------
%	ARTICLE CONTENTS
%----------------------------------------------------------------------------------------
\section*{Section 1: Introduction}
\label{sec:section1}
In this report I will go into detail about my solution to stage 1 of the Major Assessment. In stage 1,  
a connection is created between a simulated DS server and a client server. Which involves sending handshakes, 
server scouting and job scheduling. The purpose of this is to demonstrate how a real life client-server 
job scheduling would take place. In section 2, I provide a high-level system overview to help understand how 
the system would work. Section 3 goes into detail about the design of the overall framework, the constraints 
functionalities etc.,. Section 4 explains the implementation of the solution, libraries involved, technologies 
used etc., 

\subsection*{Index}
\begin{itemize}
    \item \hyperref[sec:section1]{Section 1: Introduction}
    \item \hyperref[sec:section2]{Section 2: High-Level Description}
    \item \hyperref[sec:section3]{Section 3: Design}
    \item \hyperref[sec:section4]{Section 4: Implementation}
    \item \hyperref[sec:section5]{Section 5: References}
\end{itemize}

\section*{Section 2: High-Level Description}\label{sec:section2}
\includegraphics*[scale=0.708]{Flowchart.png}

\section*{Section 3: Design}\label{sec:section3}

\begin{table}[h!]
    \centering
    \begin{tabular}{|c|c|c|c|c|}
    \hline
        Heading 1 & Heading 2 & Heading 3 & Heading 4 & \ldots \\
    \hline
    A & \$100.0 &&& \\\hline
    B & $\sim$ \$93 &&& \\\hline
    C & $10^2$ &&& \\\hline
    D & $C_2$ &&& \\\hline
    E & 85\% &&& \\\hline
    \end{tabular}
    \label{tab:my_label}
    \caption{Average Performance.}
\end{table}

\subsection*{Criticisms}
Criticism/reflection and improvement suggestions

\section*{Section 4: Implementation}\label{sec:section4}

\section*{Section 5: References}\label{sec:section5}
%----------------------------------------------------------------------------------------
%	REFERENCE LIST
%----------------------------------------------------------------------------------------
\bibliographystyle{ieeetr}
\bibliography{comp3100project}
% \printbibliography

%----------------------------------------------------------------------------------------

\end{document}
